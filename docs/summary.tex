
\documentclass[french]{article}
\usepackage[T1]{fontenc}
\usepackage[utf8]{inputenc}
\usepackage{geometry}
\geometry{verbose}
\usepackage{babel}
\addto\extrasfrench{%
   \providecommand{\og}{\leavevmode\flqq~}%
   \providecommand{\fg}{\ifdim\lastskip>\z@\unskip\fi~\frqq}%
}

\begin{document}

\title{Roguelike}

\maketitle

\section{Système et concepts :}


\subsection{Invariants}
\begin{itemize}
\item Un seul joueur par partie : toutes les autres entités sont simulées
par une intelligence artificielle.
\item Tour par tour, chaque tour étant déclenché par l'action du joueur.
\item Les cartes sont des grilles 2D ; le plateau de jeu d'une partie est
composé de plusieurs cartes qui ont des points de passages entre elles
(portes, escaliers, portails, à sens unique ou non) ; typiquement,
un donjon composé de plusieurs étages.
\item Un seul personnage par case.
\item Le joueur ne connait de la carte que ce qu'il en a exploré.
\item Le joueur ne voit que les objets et entités qui sont à sa portée.
\item Même lorsque le joueur ne les voit pas, le comportement des objets
et entités reste cohérent.
\end{itemize}

\subsection{Jeu}

Le joueur peut effectuer des actions toujours possibles (se déplacer),
utiliser des objets de son inventaire (boire une potion) ou intéragir
avec des objets extérieurs (ouvrir une porte), dans le but d'explorer
son monde, de vaincre des ennemis ou d'accomplir des quêtes.


\subsection{Temps }

Chaque action a une certaine durée.

Un tour de jeu correspond à une action du joueur. Un tour de jeu n'est
donc pas une unité de temps fixe, mais peut être plus ou moins long
dans le temps du jeu selon l'action effectuée, bien que cela soit
transparent pour le joueur.

Durant ce tour, cette action est effectuée, et le temps de l'action
s'écoule dans le jeu (les créatures susceptibles de se déplacer se
déplacent, des événements peuvent se produire, etc).

Rien ne se produit dans le jeu sans action du joueur, qui peut donc
prendre autant de temps qu'il le souhaite pour jouer, sans conséquence.


\section{Choix d'architectures et d'implémentation}


\subsection{Actions}


\subsection{Objets}


\subsubsection{Objets mutables et constants}

Les objets peuvent être séparés en deux catégories dont le traitement
sera différent : les objets qui ont un comportement propre, et les
objets \og constants \fg{}, i.e. dont l'état peut changer, mais
pas de leur propre initiative.

Les premiers agissent à chaque tour de jeu ; ils doivent donc être
parcourues à chaque tour et sont donc stocké dans une structure de
type \texttt{List}. Ils contiennent une référence sur leur emplacement
actuel. Lors de ce parcours, le moteur de jeu garde une référence
vers les objets qui sont à portée du joueur.

Les seconds sont plus statiques, mais doivent pouvoir être récupérés
rapidement depuis leur position : ils sont donc stockés dans une structure
de type \texttt{Map}.


\subsubsection{Fonctionnalités}

Une fonctionnalité d'un objet correspond à un trait dont il bénéficie.
Voir le fichier \texttt{/src/draft/traits.scala}, par exemple.


\subsubsection{Position d'un objet}

La position d'un objet peut être : 
\begin{itemize}
\item sa position sur une carte donnée – i.e. un triplet $(x,y,z)$, $z$
référencant la carte.
\item une référence vers un objet de trait \texttt{container}.
\end{itemize}

\section{Questions en suspens}
\begin{itemize}
\item Représentation de la carte ?
\end{itemize}

\section{Extensions envisagées}
\end{document}
